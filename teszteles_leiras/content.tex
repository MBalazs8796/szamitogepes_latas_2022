\section{Frissített feladatleírás}

A feladatunk egy algoritmus megtervezése, elkészítése, és tervezése, amely segítségével egy valós kamera felvételbe be lehet majd szúrni egy tetszőgeles virtuális geometriát, ami hihetően reagál a kamera mozgására, és a környezetére.
A feladat nehézsége miatt számos megszorítással, és feltétellel fogunk dolgozni, ezek a következőek:

\begin{itemize}
	\item A geometria a környezetéhez vett realtív pozíciója statikus, tahát a felvételen csak a kamera mozog
	\item A geometria nem reagál a környező fényhatásokra, nem lesz árnyékolva, és nem is vet árnyékot
	\item A felvétel előre fel lesz véve, az algoritmus nem feltétlenül lesz alkalmas valós idejű használatra
	\item A geometria kezdeti pozíciója előre, manuálisan meghatározott
	\item A geometria és a kamera közé a felvétel során nem kerül obstrukció
\end{itemize}

\section{Az algoritmus összefoglalva}

Az algoritmusunk nagyobb részre osztható:

Első lépésként meghatározzuk a videót felvevő kamera mozgását az ORB-SLAM2~\cite{mur2015orb} segítségével.
Az algoritmus részletesebb leírását az előző mérföldkő során bemutattuk, így itt eltekintünk tőle.
Az ORB-SLAM2 a videó analízise után biztosítja a végső keyframe gyűjteményét, illetve az ezekben a pillanatokban általa azonosított kamera pozíciót, ezt felhasználva próbáljuk majd a kérdéses geometriát megfelelően transzformálni.
Fontos kiemelni, hogy az ORB-SLAM2 futás közben optimalizálja a keyframek számát, azaz elvet azokat, amelyeket redundánsnak tart, azaz olyanokat, melyek túl hasonlóak egymáshoz.

A keyframe halmaz előállításához elengedhetetlen a felvevő kamera pontos kalibrációja, pontatlan kalibráció esetén az ORB-SLAM2 gyakran nem képes kellően elvégezni a szükséges pontmegfeleltetéseket, ezáltal nem lesz képes inicializálni a kamera pozícióját, ami üres kimenethez, és egy feldolgozatlan videóhoz vezet.
A kalibrációt videó alapján végeztük, így ugyanis könnyebben tudtunk nagy mennyiségű képhez jutni, illetve el tudtuk kerülni a telefonjaink videó és fényképező módjai közötti előfeldolgozási diszparitás okozta hibákat.

A keyframek meghatározása mellett fontos előfeldolgozási feladatként elhelyeztük az objektumokat az adott jelenetbe úgy, hogy végig láthatóak lehessenek.
Az elhelyezést mindig a videó első képkockáján végeztük, arra való tekintet nélkül, hogy a geometria a videó fókuszpontjában legyen.

A második főbb lépés a kérdéses geometria betöltése, illetve helyes transzformálása a kamera aktuális pozíciója alapján.
Az ORB-SLAM2 kimenetében a kamera pozíciója az inicializálás pillanatában vett helyzetének transzformációival van megadva, így járható útnak tűnik, hogy ha a geometriát a kamerára alkalmazott transzformációk inverzének vetjük alá, akkor a térben felvett relatív pozíciója statikus marad, ezáltal elérve a kívánt hatást.
Ennek a megközelítésnek egy konkrét impelentációja elérhető GitHub-on\footnote{https://github.com/ChiWeiHsiao/Match-Moving}.
Egyelőre mi is elvetjük a nem keyframe szakaszait a vizsgált videónak, ugyanis a geometria a köztes pillanatokban felvett pozíciójának meghatározása nem javítana az algoritmus pontosságán.


\section{Tesztek}

A tesztek tervezése során elsődleges szempont volt, hogy a lehető legszélesebb spektrumot fedjünk le.
A kamera mozgási sebessége, iránya, a környezet megvilágítása, a jelenetben található tárgyak színei, formái, a geometria relatív pozíciója a kamerához, és számos további potenciálisan jelentős faktort próbálunk figyelembe venni, és lefedni.
Természetesen a tesztjeink közel sem képesek a teljes probléma teret lefedni, azonban megpróbáltuk azonosítani azokat az eseteket, amik a lehető legnagyobb teret fedik le.

A tesztek során elsődleges célunk az algoritmus limitjeinek meghatározása volt, nem csupán az, hogy példákat gyűjtsünk helyes, és helytelen működésre.
A megfelelő működést pontosan úgy definiáljuk, hogy a geometria egy vizuális hibahatáron belül helyezkedik el, azaz nem feltétlenül szükséges, hogy a geometria tökéletesen stabilan a megfelelő pozícóban maradjon, ez a különböző kerekítési hibák, és szükséges közelítések miatt szinte lehetetlen.
Az helyes működés így könnyebben elérhető, ezáltal a határok viszont tágabbak, ami nehezebbé teszi a meghatározásukat.

Az elsősorban figyelembe vett paraméterek, a kamera mozgásának komplexitása, a környezet komplexitása, és a megvilágítás volt.

A kamera mozgásának komplexitását 2 fő faktor mentén határozhatjuk meg.
Az első, hogy a kezdeti pozícióból egy tetszőleges időpillanatban mennyire komplikált transzformáció során vihetnénk át.
Tehát, ha például a videó során csak eltolások történnek, különösen, ha ezek egyetlen tengelyt követnek, az kevésbé komplikált, mintha a kamera forogna 1, vagy több tengely mentén.

A második, hogy a kamera által felvett pozíció mennyire eltérő transzformáció során jöhetett létre 2 tetszőleges időpillanat között.
Ebben az esetben ha a kamera több tengely mentén forog/mozog, de egy irányba teszi azt konzisztensen, akkor azt állítjuk, hogy kevésbé komplex a mozgása, mintha például egy egyszerűbb utat járna be, de időközönként egy új véletlen irányba kezdene forogni.

A környezet komplexitása szintén potenciálisan több faktortól függ, a jelenlévő tárgyak számától, azok sarokpontjainak beazonosíthatóságától, és így tovább.
Természetes közegekben ezt a komplexitást nehéz korlátozni, hiszen nehéz inkrementálisan csökkenteni egy fa asztal mintájának összetettségét, vagy egy pohár sarkainak kerekítettségének mértékét befolyásolni, így ezt a faktort kevéssé vizsgáltuk.

Fontos kiemelni, hogy egy feladat, ami komplexebb részekkel rendelkezik nem feltétlenül lesz nehezebb, mint egy olyan, ami egyszerűbb, hiszen például minimális környezeti komplexitás esetén az ORB-SLAM nem fog tudni kellő mennyiségű pontmegfeleltetést elvégezni ahhoz, hogy megfelelően tudja követni a kamerát.

\section{Metrikák}

Annak érdekében, hogy számszerűsíteni tudjuk az algoritmusunk pontosságát 2 metrikát vezettünk be.
Mindkét metrika során a keyframeket vizsgáljuk, hiszen jelenleg csak ezeket dolgozzuk fel.

Az első során egy bináris döntést hozunk arról, hogy az adott pillanatban helyesen számítottuk-e ki a geometriára alkalmazandó transzformációt, vagy sem.
Ebben az esetben a videó keyframein egyesével végigmenve a kiértékelő meghatározza, hogy az objektum egy elfogadható hibahatáron belül helyezkedik-e el vagy sem, azaz viszonylag közel van-e ahhoz a pozícióhoz, ahol lennie kellene, ha az algoritmus helyesen működne.
A metrika több szempontból is pontatlan, hiszen nehéz objektív konzisztenciát elvárni egy emberi megfigyelőtől, abban az esetben, ha a transzformáció az esetek nagyobb részében közel helyes.
Ilyenkor előfordulhat, hogy egyes képeken nagyobb eltérést fogadunk el, mint máshol, szimplán azért, mert "természetesebben néz ki", vagy mi magunk sem tudjuk pontosan eldönteni, hogy hogyan kellene kinéznie az adott jelenetnek helyes működés esetén.
Az algoritmusunk jelenlegi verziója mellett ezek a problémák nem jelentősek, hiszen nagyon szignifikáns hibákat vét az esetek jelentős arányában, így egyelőre ez a durva metrika is használható.
Abban az esetben, ha sikerül olyan magas szofisztikációt elérni az algoritmusunk, hogya megfigyelő szubjektivitásának zavaró hatása jelentőssé válna, ez a metrika elhagyható, hiszen már elértünk egy "elég jó működést", ami után már csak mérőszámokon tudunk javítani.

A második metrika is manuális értékelésre alapszik, azonban a kimenete már számszerűsített, így jobban használható az inkrementális javulások mérésére.
A számításához egy kiválasztott keyframe halmazra, vagy akár az összes keyframere manuálisan meghatározzuk, hogy mi lenne az objektum helyes pozíciója, majd a transzformációs mátrixot, ami ehhez a pozícióhoz vezetett összehasonlítjuk a 2. mátrixnorma mentén ahhoz, amit a módszerünk határozott meg az adott képkockára.
A normák különbségét használjuk, mint pontosság.
Ebben az esetben a kézzel meghatározott pozíció pontossága egy határozott gyengepont, hiszen emberek számára is nagyon nehéz megmondani, hogy pontosan hol kellene egy objektumnak elhelyezkedni, az ez által bevezett hiba azonban az előzőhöz hasonlóan akkor lenne szignifikáns, ha a módszerünk nagyságrendekkel szofisztikáltabb lenne.
Jelenleg a távolság a "helyes" és a számított pozíció között elég nagy ahhoz, hogy az emberi hiba ne legyen komoly befolyással a pontosságra.
Abban az esetben pedig, ha egy olyan pontra jutnánk, amikor ez egy szignifikáns probléma lenne, a módszer által meghatározott pozíciót használhatnánk kiinduló pontnak, a pontos helyzet meghatározásához, hiszen ebben az esetben már csak minimális javításokra lenne szükség.

A második metrika nem csak könnyebben számszerűsíthető, és de könnyen egyszerűsíthető is, hogy csak bizonyos paraméterek változására legyen érzékeny.
Elkészíthető egy forgás, vagy eltolás invariáns verzió, ami segítségével jobban szétbontható az egyes irányokba elért fejlődés.

Mindkét metrika erősen emberi bemenetre hagyatkozik, azonban mivel ennél a problémánál a "helyes" transzformáció legegyszerűbben a-mentén definiálható, hogy mi tűnik "természetesnek" egy emberi megfigyelő számára, így ez szerintünk elkerülhetetlen.

\section{Eredmények}

\section{Potenciális javítási lehetőségek}


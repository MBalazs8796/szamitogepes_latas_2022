\section{Frissített feladatleírás}

A feladatunk egy algoritmus megtervezése, elkészítése, és tervezése, amely segítségével egy valós kamera felvételbe be lehet majd szúrni egy tetszőgeles virtuális geometriát, ami hihetően reagál a kamera mozgására, és a környezetére.
A feladat nehézsége miatt számos megszorítással, és feltétellel fogunk dolgozni, ezek a következőek:

\begin{itemize}
	\item A geometria a környezetéhez vett realtív pozíciója statikus, tahát a felvételen csak a kamera mozog
	\item A geometria nem reagál a környező fényhatásokra, nem lesz árnyékolva, és nem is vet árnyékot
	\item A felvétel előre fel lesz véve, az algoritmus nem feltétlenül lesz alkalmas valós idejű használatra
	\item A geometria kezdeti pozíciója előre, manuálisan meghatározott
	\item A geometria és a kamera közé a felvétel során nem kerül obstrukció
\end{itemize}

\section{Az algoritmus összefoglalva}

Az algoritmusunk nagyobb részre osztható:

Első lépésként meghatározzuk a videót felvevő kamera mozgását az ORB-SLAM2~\cite{mur2015orb} segítségével.
Az algoritmus részletesebb leírását az előző mérföldkő során bemutattuk, így itt eltekintünk tőle.
Az ORB-SLAM2 a videó analízise után biztosítja a végső keyframe gyűjteményét, illetve az ezekben a pillanatokban általa azonosított kamera pozíciót, ezt felhasználva próbáljuk majd a kérdéses geometriát megfelelően transzformálni.
Fontos kiemelni, hogy az ORB-SLAM2 futás közben optimalizálja a keyframek számát, azaz elvet azokat, amelyeket redundánsnak tart, azaz olyanokat, melyek túl hasonlóak egymáshoz.

A keyframe halmaz előállításához elengedhetetlen a felvevő kamera pontos kalibrációja, pontatlan kalibráció esetén az ORB-SLAM2 gyakran nem képes kellően elvégezni a szükséges pontmegfeleltetéseket, ezáltal nem lesz képes inicializálni a kamera pozícióját, ami üres kimenethez, és egy feldolgozatlan videóhoz vezet.
A kalibrációt videó alapján végeztük, így ugyanis könnyebben tudtunk nagy mennyiségű képhez jutni, illetve el tudtuk kerülni a telefonjaink videó és fényképező módjai közötti előfeldolgozási diszparitás okozta hibákat.

A keyframek meghatározása mellett fontos előfeldolgozási feladatként elhelyeztük az objektumokat az adott jelenetbe úgy, hogy végig láthatóak lehessenek.
Az elhelyezést mindig a videó első képkockáján végeztük, arra való tekintet nélkül, hogy a geometria a videó fókuszpontjában legyen.

A második főbb lépés a kérdéses geometria betöltése, illetve helyes transzformálása a kamera aktuális pozíciója alapján.
Az ORB-SLAM2 kimenetében a kamera pozíciója az inicializálás pillanatában vett helyzetének transzformációival van megadva, így járható útnak tűnik, hogy ha a geometriát a kamerára alkalmazott transzformációk inverzének vetjük alá, akkor a térben felvett relatív pozíciója statikus marad, ezáltal elérve a kívánt hatást.
Ennek a megközelítésnek egy konkrét impelentációja elérhető GitHub-on\footnote{https://github.com/ChiWeiHsiao/Match-Moving}.
Egyelőre mi is elvetjük a nem keyframe szakaszait a vizsgált videónak, ugyanis a geometria a köztes pillanatokban felvett pozíciójának meghatározása nem javítana az algoritmus pontosságán.


\section{Tesztek}

\section{Metrikák}

\section{Eredmények}

\section{Potenciális javítási lehetőségek}


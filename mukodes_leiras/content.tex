\section{Feladatleírás}

Ide kellene írni a feladat pontos megfogalmazását, megszorításokkal együtt

\section{Megoldási lehetőségek}

Ide gondoltam, felírni, hogy Zita miket talált, miért lettek volna jók, és miért nem ezeket választottuk végül.

\section{ORB-SLAM}

\subsection{ORB-SLAM bevezető}

Az ORB-SLAM~\cite{7219438} egy monokulár SLAM rendszer, ami valós időben képes működni közel bármilyen környezetben. 
A SLAM (Simultaneous localization and mapping) rendszerek arra használhatók, hogy létrehozzunk és frissítsünk egy 3D-s térképet egy ismeretlen környezetről, miközben folyamatosan követjük a kamera pozícióját. 
Ebből is látható, hogy az általunk kitűzött célra tökéletes választás az ORB-SLAM.

Magának a rendszernek több verziója is létezik, technikai nehézségekből adódóan mi az ORB-SLAM 2-vel dolgozunk. 
A 2. és 3. verzió között követéspontossági eltérések vannak csak, az alapvető működésük egyforma. 
Amennyiben a fejlesztés során problémák adódnának a pontosságból, akkor az utolsó, javítási szakaszban át tudunk térni a 3. verzióra.

\subsection{ORB}

\subsubsection{ORB bevezető}

Ide gondoltam az ORB leírását, mit tud, miért erős stb. kb. addig, hogy miket használ.

\subsubsection{FAST}

Ide jönne Máté része, ahol kicsit részletesebben beszélünk ezekről az algoritmusokról

\subsubsection{BRIEF}

-- UA mint FAST --

\subsubsection{ORB részletek}

Itt visszatérnénk az ORB-ra leírom majd, hogy mi az oFast és az rBrief, ami véglegesíti, hogy miért annyira jó

\subsection{ORB-SLAM részletek}

Az ORB-SLAM központi algoritmusa az úgynevezett Bundle Adjustment (BA). 
Ennek a lényege, hogy egyidőben történik a tér 3D geometriájának, a relatív mozgás paramétereinek és a kamera optikai tulajdonságainak finomítása.
Több módszer is használta a BA-t korábban, de az ORB-SLAM ezen felül több dolgot is csinál, többek között: 
\begin{itemize}
	\item Ugyanazokat a jellemzőket használja az összes feladathoz
	\item Covisibility gráf segítségével nagy környezetekben is képes valósidejű működésre
	\item Esszenciális gráf segítségével valós időben képes lezárni a köröket
\end{itemize}

\subsubsection{Az ORB-SLAM definíciói}

\textit{Térkép pont.} Minden $p_i$ térkép pont tartalmazza: az $X_{w, i}$ pozícióját a 3D-s térben, a megfigyelés $n_i$ irányát, a reprezentáns $D_i$ ORB leírót és a $d_max$ és $d_min$ távolságokat, amikből a pont megfigyelhető.

\textit{Keyframe.} Minden $K_i$ keyframe tartalmazza: a $T_iw$ kamera pózt, ami egy merev test transzformáció, ami a pontot a világ koordináta rendszerből a kamera koordináta rendszerébe transzformálja, a kamera paramétereit és minden ORB jellemzőt, amit a képkockából ki lehet nyerni. 

\textit{Covisibility gráf.} A Covisibility gráf egy súlyozott, irányítatlan gráf ami a keyframe-ek közötti láthatóságot reprezentálja. 
Minden csúcs egy keyframe, és két csúcs között akkor létezik él, ha mindkét keyframe tartalmaznak közös térkép pontokat (legalább 15-öt). 
Az él súlya pedig a közös képpontok száma ($\theta$).

\textit{Esszenciális gráf.} Az esszenciális gráf a covisibility gráfból készül: a csúcsok megegyeznek, viszont sokkal kevesebb éle van. 
Az esszenciális gráf a covisibility gráf feszítőfájából, a kört záró élekből és a covisibility gráf azon éleiből áll, ahol $\theta_{min} = 100$.

\subsubsection{Működés összefoglalva}
Az ORB-SLAM egyszerre három szálon fut: az egyiken történik a követés, a másikon a lokális mapping, a harmadikon pedig a kör bezárása.
A követés feladata, hogy a kamerát lokalizálja minden egyes képkockán, és eldöntse, hogy mikor szükséges új keyframe beszúrása.
A lokális mapping feldolgozza az új keyframe-eket, és lokális BA segítségével optimális módon rekonstruálja a környezetet.
A kör bezáró szál pedig minden új keyframe beszúrásakor kört keres az esszenciális gráfban, és amennyiben talál, akkor összeolvasztja a duplikált pontokat.

\subsubsection{Követés}

\textit{Automatikus térkép inicializálás.} 


\section{Mapping}

Itt fogom majd összeírni, hogy az ORBSLAM outputból hogyan fogunk a match movehoz jutni.